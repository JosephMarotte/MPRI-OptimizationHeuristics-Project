\documentclass[12pt]{article}

\usepackage[utf8]{inputenc}
\usepackage[T1]{fontenc}
\usepackage{amsmath}
%\usepackage{amsfonts}
%\usepackage{geometry}
%\usepackage{graphicx}
%\usepackage{url}
%\usepackage{amsthm}
%\usepackage{float}
\usepackage{mathpazo}

\usepackage
[
    a4paper,
    left=1.5cm,
    right=2.5cm,
    top=1.5cm,
    bottom=2cm
]{geometry}

\usepackage[parfill]{parskip}    % Activate to begin paragraphs with an empty line rather than an indent

\usepackage{amsthm}
\theoremstyle{definition}
\newtheorem*{definition}{Definition}
\newtheorem*{notation}{Notation}
\newtheorem*{example}{Example}
\newtheorem*{problem}{Problem}
\theoremstyle{plain}
\newtheorem*{lemma}{Lemma}
\newtheorem*{theorem}{Theorem}
\newtheorem*{proposition}{Proposition}
\theoremstyle{remark}
\newtheorem*{remark}{Remark}

% Maths operators 
\DeclareMathOperator{\Tr}{Tr}
\DeclareMathOperator{\Sep}{Sep}
\DeclareMathOperator{\Conv}{Conv}

% My commands
\newcommand{\myequiv}[1]{\underset{#1}{\sim}}
\newcommand\mydef{\mathrel{\overset{\makebox[0pt]{\mbox{\normalfont\tiny\sffamily def}}}{=}}}

\title{Analysis of local-search heuristics for Mastermind with $n$ colors}
\author{}
\date{}
\begin{document}

\maketitle

Here we prove tight lower bounds and upper bounds for RLS, and almost tight lower bounds and
upper bounds for $(1+1)\text{-EA}$.

\begin{notation}
    \begin{itemize}
        \item[]
        \item $\log$ denotes the logarithm in base $e$.
        \item $[n]\mydef \{1,\ldots,n\}$
        \item $H_n\mydef \sum_{i=1}^n \frac{1}{i}$
    \end{itemize}
\end{notation}

\begin{definition}[$\text{MM}_n$]
    As all our algorithms will be unbiased, we consider the objective function $f:[n]^n\to \mathbf{R}_+$ defined by:

    $$f(x)\mydef \sum_{i=1}^{n} \mathbf{1}_{x_i=0}$$
\end{definition}

\begin{remark}
    For RLS, we initialize $x\in_R [n]^n$ and at each step we choose an index
    $i\in_R [n]$ and a shift $s\in_R [n-1]$.
\end{remark}


\begin{theorem}
    $$\mathbf{E}[\mathcal{T}(\text{RLS},\text{MM}_n)]\myequiv{n\to\infty} n^2 \log n$$
\end{theorem}

\begin{proof}
    Let $x^0, \ldots x^t$ be the successive queries of RLS.
    Let $T_i\mydef \min\{t\mid f(x^t)\ge i\}$. Note that for all $i\ge i_0$, the
    variable $T_{i+1}-T_i$ conditioned on $f(x^0)=i_0$ follows a geometric distribution
    with parameter $\frac{n-i}{n}\frac{1}{n-1}$. So:

    $$\mathbf{E}[T_{i+1}-T_i\mid f(x^0)=i_0]=\frac{n(n-1)}{n-i}$$
    
    And it follows that:
    
    $$\mathbf{E}[T_{n}\mid f(x^0)=i_0]=n(n-1)\sum_{i=i_0}^{n-1} \frac{1}{n-i}=n(n-1)H_{n-i_0}$$

    If $c_n\mydef \sqrt{n\log n}$, by additive Chernoff bounds:

    $$\mathbf{P}[f(x^0)> 1+c_n]\le \exp(-2\log n)=\frac{1}{n^2}$$

    So:

    $$\mathbf{E}[T_n]=n(n-1)\sum_{i_0\le 1+c_n} H_{n-i_0} \mathbf{P}[f(x^0)=i_0]+o(1)$$

    Finally:

    $$\left(1-\frac{1}{n^2}\right)\log(n-1-c_n)\le \sum_{i_0\le 1+c_n} H_{n-i_0} \mathbf{P}[f(x^0)=i_0]\le 1+\log n$$
    
    Hence $\mathbf{E}[T_n]\myequiv{n\to\infty} n^2\log n$.\qedhere

\end{proof}

\begin{remark}
    For $(1+1)\text{-EA}$, we consider $p=\frac{1}{n}$.
\end{remark}

\begin{theorem}
    $$\mathbf{E}[\mathcal{T}((1+1)\text{-EA},\text{MM}_n)]\le e n^2 (\log n+1)$$
\end{theorem}

\begin{proof}
    We apply the fitness local method. Let $x$ be the current state of the variable and $y$ the transformed one. 

    $$\mathbf{P}[f(y)>i\mid f(x)=i]\ge (n-i)\left(1-\frac{1}{n}\right)^{n-1} \frac{1}{n^2}\ge \frac{n-i}{e n^2}$$

    So:

    $$\mathbf{E}[\mathcal{T}((1+1)\text{-EA},\text{MM}_n)]\le e n^2 H_n\le e n^2(\log n+1)$$\qedhere
\end{proof}

\begin{theorem}
    $$\mathbf{E}[\mathcal{T}((1+1)\text{-EA},\text{MM}_n)]\ge n^2 \log n+o(n^2\log n)$$
\end{theorem}

\begin{proof}
    Let $X_{j,t}$ be the indicator of the event ``the $j$-th position is incorrect in $x^0$ and zero has
    never been drawn out for this position in the first $t$ iterations''. Let:

    $$t_n\mydef \left(1-\frac{\log \log n}{\log n}\right)(n^2-1)\log n$$

    Then at this time:

    $$\mathbf{P}[X_{j,t_n}=1]=
    \left(1-\frac{1}{n}\right)\left(1-\frac{1}{n^2}\right)^{\left(1-\frac{\log\log n}{\log n}\right)(n^2-1)\log n}\ge \frac{\log n}{2n}
    $$

    We deduce:

    $$\mathbf{E}\left[\sum_{j=1}^n X_{j,t_n}\right] \ge \frac{\log n}{2}$$

    And by multiplicative Chernoff bounds:

    $$\mathbf{P}\left[\sum_{j=1}^n X_{j,t_n}\le \frac{\log n}{4}\right] \le \exp\left(-\frac{\log n}{16}\right)=o(1)$$

    If $T$ denotes the time at which we find the optimal solution:

    $$\mathbf{P}[T\le t_n]\le \mathbf{P}\left[\sum_{j=1}^n X_{j,t_n} \le \frac{\log n}{4}\right]=o(1)$$

    In the end:

    $$\mathbf{E}[T]\ge \mathbf{P}[T>t_n] t_n=(1-o(1))\left(1-\frac{\log \log n}{\log n}\right)(n^2-1)\log n\sim n^2\log n\qedhere$$
\end{proof}

\begin{remark}
    We are in a very different regime from what happens for $f:\{0,1\}^{n \log_2 n}\to \mathbb{R}_+$. Getting yes/no
    answers from groups of $\log_2 n$ pieces together somehow results in an additional $n$ factor in the efficiency of $(1+1)\text{-EA}$.
\end{remark}

We now give an upper bound for Erdos-Renyi method:

\begin{theorem}
    For all $\epsilon>0$:

    $$\mathbf{E}[\mathcal{T}(\text{ER}, \text{MM}_n)]\le (3+\epsilon) en\log n+o_\epsilon(1)$$
\end{theorem}

\begin{proof}
    The probability that the color $c>0$ at position $1\le p\le n$
    was not chosen in any sample $x^i$ such that $f(x^i)=0$ and $1\le i\le t$
    is upper bounded by:

    $$\left(1-\left(1-\frac{1}{n}\right)^{n-1}\frac{1}{n}\right)^t\le \exp\left(-t\left(1-\frac{1}{n}\right)^{n-1}\frac{1}{n}\right)\le\exp\left(-\frac{t}{en}\right)$$

    Let $T$ be the first point in time when all nonzero colors at every position have
    been chosen in a sample that evaluated to zero. Then by a union bound:

    $$\mathbf{P}[T\ge t]\le n^2 \exp\left(-\frac{t}{en}\right)$$

    Let $t_n\mydef (3+\epsilon)en\log n$. Then:

    $$\mathbf{E}[T]\le \sum_{t=1}^{t_n} \mathbf{P}[T\ge t] +\sum_{t>t_n} \mathbf{P}[T\ge t]\le t_n+\sum_{t>t_n} \frac{1}{n^{1+\epsilon}}\le t_n+o_{\epsilon}(1)$$
\end{proof}

\end{document}
